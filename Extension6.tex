\documentclass[12pt]{amsart}

\usepackage[margin=1in]{geometry}
\usepackage{paralist}
\usepackage{amssymb}
\usepackage{amsmath}
\usepackage{amsthm}
\usepackage{braket}
\usepackage{mathtools}
\usepackage{soul}

\newcommand{\bbR}{\mathbb{R}}
\newcommand{\bbN}{\mathbb{N}}
\newcommand{\bbZ}{\mathbb{Z}}
\newcommand{\bbQ}{\mathbb{Q}}
\newcommand{\bbP}{\mathbb{P}}
\newcommand{\suchthat}{\operatorname{\thinspace s.t. \thinspace}}
\newcommand{\ie}{\operatorname{i.e.}}
\newcommand{\LL}{\operatorname{LL}}

\theoremstyle{plain}
\newtheorem*{prop}{Proposition}
\newtheorem*{thm}{Theorem}
\newtheorem*{cor}{Corollary}
\newtheorem*{claim}{Claim}
\newtheorem*{axm}{Axiom}
\newtheorem*{lem}{Lemma}



\theoremstyle{remark}
\newtheorem*{rmk}{Remark}

\theoremstyle{definition}
\newtheorem*{define}{Definition}
\newtheorem*{eg}{Example}

\title{Analysis I Extension Lecture 6\\Construction of $\bbR$ through $\bbQ$}
\author{Asilata Bapat}

\begin{document}

\maketitle
\setuldepth{abc}

Henceforth we write rationals as usual: $[(-3,5)] \rightarrow -\frac{3}{5}$.
\begin{define}
	A \ul{sequence} $s$ of numbers in $\bbQ$ is a function
	\begin{equation*}
		s: \bbN  \mapsto \bbQ
	\end{equation*}
\end{define}

\begin{define}
	A sequence $s$ of numbers in $\bbQ$ is called \ul{convergent to $q\in \bbQ$} if for every $n \in \bbN$ where $n \neq 0$, there is $m\in \bbN$ such that for each $M > m$ we have 
	\begin{equation*}
		d(a_M, q) < \frac{1}{n}
	\end{equation*}
	Especially, we say that a sequence $s$ is \ul{null} if $s \rightarrow 0$.
\end{define}

\begin{define}
	A sequence $a$ of $\bbQ$ is called \ul{Cauchy} if for every $n \in \bbN$ where $n \neq 0$, there is $m \in \bbN$ such that for every $n \in \bbN$ where $n \neq 0$, there is a $m \in \bbN$ such that for every $P,M > m$, we have 
	\begin{equation*}
		d(a_P, a_Q) < \frac{1}{n}
	\end{equation*}
\end{define}

\begin{thm}
	If a sequence $s_n$ converges to $q\in \bbQ$, then  it is Cauchy.
\end{thm}

\begin{prop}
	There exists Cauchy sequences in $\bbQ$ that are not convergent.
\end{prop}

\par
We would like a enlargement of $\bbQ$, in which $Cauchy \iff convergent$
\begin{define}
	We will say a field $F$ is \ul{complete} if every Cauchy sequence converge.
\end{define}

\ul{Idea}: Given any $q\in \bbQ$, say $q = \frac{1}{2}$; consider the sequence $(q,q,q,\cdots)$. This is a Cauchy sequence with limit $q$.
Also, $S_n = (q + \frac{1}{n})$, then $s_n \rightarrow q$.
\newline
Now let $s_n = 3,3.1,3.14,3.141,3.1415,\cdots$. This is a Cauchy sequence, but it probably does not converge in $\bbQ$. But we \ul{want} it to converge.

\par
Let 
\begin{equation*}
	R = \Set{\mbox{all Cauchy sequences of } \bbQ}
\end{equation*}

and let $\bbP(\bbN \times \bbQ)$ be the power set of $\bbN \times \bbQ = \mbox{all relations } \bbN \rightarrow \bbQ$. We look at 
\begin{equation*}
	\Set{F \in \bbP(\bbN \times \bbQ) | F \mbox{ is a function that defines a Cauchy sequence}}
\end{equation*}

We want all element of $R$ to be the representative for its ``limit'', even if it does not exist. However there is a problem:
\newline
$(1,2,2,2,\cdots)$ and $(2,2,2,2,\cdots)$ go to the same limit, but they are different sequences. The way to deal with it is to include equivalence relations. We say two sequence $(a_n)$ and $(b_n)$ are \ul{equivalent } (written as $(a_n) \sim (b_n)$) if their difference $(a_n - b_n)$ converge to $0$. 

\begin{define}
	We define real number $\bbR$ as
	\begin{equation*}
		\bbR \vcentcolon = R \bigg/\sim = \mbox{equivalence classes under this relation}
	\end{equation*}
\end{define}

\begin{lem}
	Every Cauchy sequence is bounded. This means there is some $M \suchthat |a_n| leqslant M$.  
\end{lem}
\begin{lem} Every Cauchy sequence $(a_n)$ that does not converge to $0$ is bounded away from $0$. Explicitly, this means that $\exists \varepsilon > 0 \mbox{ and } N \in \bbN \suchthat \forall n \in \bbN \mbox{ and } n > N$, $|a_n| > \varepsilon$  
\end{lem}

\begin{prop}
	$\sim$ is an equivalence relation.
\end{prop}
\begin{proof}
\hfill
\newline
	\begin{enumerate}
		\item (Reflexivity)
			$(a_n) \sim (a_n)$ because $(a_n - a_n)$ is the zero sequence.
		\item (Symmetry)
		If $(a_n) \sim (b_n)$ then $(a_n - b_n) \rightarrow 0$. From the definition, $(b_n - a_n) \rightarrow -0 = 0$. So $(b_n)\sim (a_n)$
		\item (Transitivity)
			Say $(a_n) \sim (b_n)$ and $(b_n) \sim (c_n)$. Then $\forall \varepsilon > 0$, $ \exists N_1 \in \bbN $ such that
			\begin{itemize}[-]
				\item 
					$\forall n > N_1$ we have $|a_n - b_n| < \frac{\varepsilon}{2}$.
				\item
					$\forall \varepsilon > 0$, $\exists N_2 \in \bbN \suchthat \forall n > N_2$, we have $|b_n - c_n| < \frac{\varepsilon}{2}$
			\end{itemize}
			Now let $N = \max\{N_1, N_2\}$. So if $n > N$, then 
			\begin{equation*}
				|a_n - c_n| \leqslant |a_n - b_n| + |b_n - c_n| < \frac{\varepsilon}{2} + \frac{\varepsilon}{2} = \varepsilon
			\end{equation*}
			Since $\varepsilon$ is arbitrary, we see $(a_n - c_n) \rightarrow 0$. So $(a_n) \sim (c_n)$.
	\end{enumerate}
\end{proof}

\par
We now show that $\bbR$ is well-defined.
\begin{proof}
		$\bbQ$ is embedded into $\bbR$($\bbQ \xhookrightarrow[]{\quad i\quad} \bbR$). That is, given any rational number $q$ we can construct a sequence converging to $q$ as follows:
		\begin{equation*}
			q \mapsto [(q,q,q,\dots)]
		\end{equation*}
		If $p \neq q$, $i(p) \neq i(q)$ then $i(p) \neq i(q)$ because $[(p - q, p - q,\dots)] \neq [(0,0,0,\dots)]$
\end{proof}

Particular, in $\bbR$ we have a $0 = [(0,0,0,\dots)]$ and a $1 = [(1,1,1,1,\dots)]$, as shown above, therefore $0 \neq 1$.

\par
Properties of $\bbR$.[(1)]
\begin{enumerate}
	\item (Addition)
		We say that: 
		\begin{equation*}
			[(a_n)] + [(b_n)] \vcentcolon = [(a_n+b_n)]
		\end{equation*}
		Then addition is well-defined, commutative, and associative. Also $[(a_n)] + 0 = [(a_n)]$
	\item (Multiplication)
		We say 
		\begin{equation*}
			[(a_n)]\cdot[(b_n)] = [(a_nb_n)]
		\end{equation*}
		\begin{claim}
			Multiplication of $\bbR$ is well defined.
		\end{claim}
		\begin{proof}
			Suppose that $[(a_n)] = [(c_n)]$ and $[(b_n)] = [(d_n)]$. We will show that $[(a_nb_n) = [(c_nd_n)]]$, meaning that the sequence $(a_nb_n - c_nd_n) \rightarrow 0$.
			\newline
			We know 
			\begin{equation*}
				(a_n - c_n) \rightarrow 0 \mbox{ and } (b_n - d_n) \rightarrow 0
			\end{equation*}
			Further, these are Cauchy sequences. Recall that if $(s_n)$ is Cauchy, then $\forall \varepsilon > 0$, $\exists N \in\bbN \suchthat \forall m,n > N$, we have $|s_m - s_n| < \varepsilon$
			\newline
			Fix some $\varepsilon$, the corresponding $N$, and some $n > N$. Then $\forall m > N$, we have
			\begin{equation*}
			s_m \leqslant |s_m - s_n| + |s_n| \leqslant \varepsilon + |s_n|
			\end{equation*}
			So the sequence is bonded above by the max of $\Set{|s_1|, |s_2|, \dots,|s_N|,\dots,|s_{n_1}|,|s_n| + \varepsilon}$.
			Consider 
			\begin{equation*}
			\begin{split}
			|a_n b_n - c_n d_n| &= |a_n b_n - b_n c_n + b_n c_n - c_n d_n| = |b_n(a_n - c_n) + c_n(b_n - d_n)|\\
		&\leqslant |b_n||a_n - c_n| + |c_n||b_n - d_n|
			\end{split}
			\end{equation*}
			Let $M$ be a positive upper bound for both $b_n$ and $c_n$. Then $\forall \varepsilon > 0$, . 
			\begin{itemize}[-]
				\item $\exists N_1 \suchthat$ if $n > N_1$, then $|a_n - c_n| < \frac{\varepsilon}{2M}$
				\item $\exists N_2 \suchthat$ if $n > N_2$, then $|b_n - d_n| < \frac{\varepsilon}{2M}$
			\end{itemize}
			So if $N = \max\Set{N_1, N_2}$ and $n > N$, then 
			\begin{equation*}
				|a_n b_n - c_n d_n| < M\cdot \frac{\varepsilon}{2M} + M \cdot \frac{\varepsilon}{2M} = \varepsilon
			\end{equation*}
			So $|a_n - b_n| \rightarrow 0$, indicating $(a_n b_n) \sim (c_n d_n)$.
			\newline
			SO multiplication is well-defined.
			\newline
			Further, multiplication distributes  over addition; and $[(a_n)] \cdot 1 = [(a_n)]$.
		\item (Division)
			If $[(a_n)] \neq 0$, then $\exists$ some $(b_n) \suchthat$ $[(a_n b_n)] = 1$.
			\begin{proof}
				Suppose $[(a_n)] \neq 0$, which means that $(a_n)$ is Cauchy but \ul{does not converge} to $0$. So, by definition, $\exists \varepsilon > 0 \suchthat \forall N \in \bbN, \exists n > N \suchthat |a_n| > \varepsilon$
				Now fix an $\varepsilon > 0$. So $\exists N \in bbN \suchthat \forall m,n > N$, we have $|a_m - a_n| < \frac{\varepsilon}{2}$.
				\newline
				Fix $n$ to be some natural number larger than $N, \suchthat |a_n| > \varepsilon$. Observe that 
				\begin{equation*}
					|a_n| < |a_n - a_m| + |a_m| < \frac{\varepsilon}{2}
				\end{equation*}
				This means that $\exists \varepsilon > 0$<++>
			\end{proof}<++>
		\end{proof}
\end{enumerate}<++>


\end{document}
