\documentclass[12pt]{amsart}

\usepackage[margin=1in]{geometry}
\usepackage{paralist}
\usepackage{amssymb}
\usepackage{amsmath}
\usepackage{amsthm}
\usepackage{braket}
\usepackage{mathtools}
\usepackage{soul}

\newcommand{\bbR}{\mathbb{R}}
\newcommand{\bbN}{\mathbb{N}}
\newcommand{\bbZ}{\mathbb{Z}}
\newcommand{\bbQ}{\mathbb{Q}}
\newcommand{\suchthat}{\operatorname{s.t.}}
\newcommand{\ie}{\operatorname{i.e.}}
\newcommand{\LL}{\operatorname{LL}}

\theoremstyle{plain}
\newtheorem*{prop}{Proposition}
\newtheorem*{thm}{Theorem}
\newtheorem*{cor}{Corollary}
\newtheorem*{claim}{Claim}
\newtheorem*{axm}{Axiom}
\newtheorem{AXM}{Axiom}
\newtheorem*{lem}{Lemma}

\theoremstyle{remark}
\newtheorem*{rmk}{Remark}

\theoremstyle{definition}
\newtheorem*{define}{Definition}
\newtheorem*{eg}{Example}

\title{Analysis I Extension Lecture 3\\Arithmetic on $\bbR$}
\author{Asilata Bapat}

\begin{document}
\maketitle
\setuldepth{relation}

How can we make sense of ``$1+2 = 3$'' if $1 = \Set{\varnothing}$ and $2 = \Set{\varnothing, \Set{\varnothing}}$?
The strategy is to define a function of ``addition of $m$'' for every $m \in \bbN$. 

\begin{thm}
	For every $m \in \bbN$, there exists a function $S_m \subseteq \bbN \times \bbN$ such that 
	\begin{enumerate}[(1)]
		\item $S_m(0) = m$
		\item $\forall n \in \bbN$, $S_m(n^+) = (S_m(n))^+$
	\end{enumerate}
\end{thm}

\begin{proof}
	First we show that there exists at least one \ul{relation} $S_m' \subseteq \bbN \times \bbN$ with the ``right'' properties. Then we show that $S_m'$ is a function.
	Define 
	\begin{equation*}
		S_m \vcentcolon = \Set{R \subseteq \bbN \times \bbN | (0, m) \in R \text{ and if } (n,x) \in R \text{ then } (n^+, m^+)\in R}
	\end{equation*}
	$S_m \neq \varnothing$ because $\bbN \times \bbN \in S_m$. Set
	\begin{equation*}
		S_m' \vcentcolon = \bigcap\limits_{R\in S_m}R
	\end{equation*}
	which is the smallest relation with these properties. 
	\newline
	We will show that $S_m'$ is indeed a function. It has the required properties by construction.
	\begin{enumerate}[(1)]
	\item $\operatorname{Domain}(S_m') = \bbN$.
		\newline
		Let $p(n)$ be true if $\exists x \suchthat (n, x)\in S_m'$. $p(0)$ is true because $(0, m) \in S_m'$. 
		\newline
		Suppose $p(n)$ is true if $\exists x \in \bbN \suchthat (n, x) \in S_m'$. So $\forall R \in S_m$, $(n, x) \in R \implies (n^+, x^+) \in R \implies (n^+, x^+) \in S_m' \implies p(n^+)$ is true. So by induction, $\operatorname{Domain}(S_m') = \bbN$.
	\item We want to show that if $(n, x), (n, y)$ are both in $S_m'$, then $x = y$.
	\newline
	Let $p(n)$ be true if $\forall r \in n, \exists! x \in \bbN \suchthat (r, x) \in S_m'$. Note that $p(0)$ is vacuously true.
	\newline
	Suppose $p(n)$ is true, we want to show $p(n^+)$ is also true. Let $r \in n^+ = n \cup \Set{n}$. This means that $r = n $ or $ r \in n$.
	\newline
	If $r \in n$, we already know the statement is true; If $r = n$. Suppose that $(n, a) \in S_m'$ and $(n, b) \in S_m'$.
	\begin{claim}
		At most one of $(n,a)$ and $(n,b)$ can have a ``predecessor'' in $S_m'$. 
	\end{claim}
	Suppose $(p_1, x_1) \in S_m'$ and $(p_2, x_2) \in S_m' \suchthat (p_1^+, n_1^+) = (n, a)$ and $(p_2^+, n_2^+) = (n, b)$. Since $p_1^+ = p_2^+ = n$, we have $p_1 = p_2$ and $p_1 \in n \implies (p_1,x_1) \in S_m'$ and $(p_2, x_2) \in S_m'$, which implies $x_1 = x_2$.
	\newline
	So, $a = x_1^+$ and $b = x_2^+ \implies a = b$. If $a \neq b$, either $\nexists (p_1^+, x_1^+) = (n, a)$ or $\nexists (p_2^+, x_2^+) = (n, b)$.
	\newline
	Suppose (WLOG) $\nexists p_1 \suchthat (p_1^+, x_1^+) = (n, a)$ and $(p_1, x1) \in S_m'$, then we can \ul{remove} $(n, a)$ from $S_m'$ and we get a smaller relation with the same properties! This cannot happen.
	\newline
	So, $\exists! a \suchthat (n, a) \S_m'$, and so $p(n^+)$ also is true. Thi
	\end{enumerate}
\end{proof}<++>
\end{document}
