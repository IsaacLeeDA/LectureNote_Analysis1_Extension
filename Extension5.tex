\documentclass[12pt]{amsart}

\usepackage[margin=1in]{geometry}
\usepackage{paralist}
\usepackage{amssymb}
\usepackage{amsmath}
\usepackage{amsthm}
\usepackage{braket}
\usepackage{mathtools}
\usepackage{soul}

\newcommand{\bbR}{\mathbb{R}}
\newcommand{\bbN}{\mathbb{N}}
\newcommand{\bbZ}{\mathbb{Z}}
\newcommand{\bbQ}{\mathbb{Q}}
\newcommand{\bbP}{\mathbb{P}}
\newcommand{\suchthat}{\operatorname{s.t.}}
\newcommand{\ie}{\operatorname{i.e.}}
\newcommand{\LL}{\operatorname{LL}}

\theoremstyle{plain}
\newtheorem*{prop}{Proposition}
\newtheorem*{thm}{Theorem}
\newtheorem*{cor}{Corollary}
\newtheorem*{claim}{Claim}
\newtheorem*{axm}{Axiom}
\newtheorem*{lem}{Lemma}



\theoremstyle{remark}
\newtheorem*{rmk}{Remark}

\theoremstyle{definition}
\newtheorem*{define}{Definition}
\newtheorem*{eg}{Example}

\title{Analysis I Extension Lecture 5\\Construction of the Rational $\bbR$}

\author{Asilata Bapat}

\begin{document}

\maketitle
\setuldepth{document}

The construction of the rational $\bbR$ is very similar to that of of the Integers $\bbZ$.
\begin{define}
	We let $Q$ be equivalence classes of ordered pairs of (certain) integers:
	\begin{equation*}
		Q \vcentcolon = \Set{(a,b) \in \bbZ \times \bbZ | b \neq 0}
	\end{equation*}
\end{define}

\begin{rmk}
	Remember, $0 = [(0,0)] \in \bbZ\cdots$
\end{rmk}

\begin{define}
	$R\in Q \times Q$ is defined as follows:
	\begin{equation*}
		R = \Set{((a,b), (c,d)) \in Q \times Q | ad = bc}
	\end{equation*}
\end{define}
\begin{claim}
	$R$ is an equivalence relation
\end{claim}

\begin{enumerate}[(1)]
	\item (Order)
		Say $[(a,b)]\in Q$. We say that $[(a,b)] > 0$ if 
		\begin{equation*}
			b > 0 \mbox{ and } a > 0 \mbox{, \ul{or}} b < 0 \mbox{ and } a < 0
		\end{equation*}
		This is a total order on $Q$ and satisfies trichotomy.
	\item (Distance)
		Write $Q^{\geq 0}$ to be the set of rationals that are either $0$ or positive. We have  function $d: Q \times Q \mapsto Q^{\geq 0}$, defined as 
		\begin{equation*}
			d([(a,b)],[(c,d)]) = 
			\begin{case}
				
			\end{case}<++>
		\end{equation*}<++>
\end{enumerate}<++>

\end{document}
