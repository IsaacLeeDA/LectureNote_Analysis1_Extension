\documentclass[12pt]{amsart}

\usepackage[margin=1in]{geometry}
\usepackage{paralist}
\usepackage{amssymb}
\usepackage{amsmath}
\usepackage{amsthm}
\usepackage{braket}
\usepackage{mathtools}
\usepackage{soul}
\usepackage{parskip}

\newcommand{\bbR}{\mathbb{R}}
\newcommand{\bbN}{\mathbb{N}}
\newcommand{\bbZ}{\mathbb{Z}}
\newcommand{\bbQ}{\mathbb{Q}}
\newcommand{\bbP}{\mathbb{P}}
\newcommand{\bbC}{\mathbb{C}}
\newcommand{\suchthat}{\operatorname{\thinspace s.t. \thinspace}}
\newcommand{\ie}{\operatorname{\thinspace i.e. \thinspace}}
\newcommand{\LL}{\operatorname{LL}}

\theoremstyle{plain}
\newtheorem*{prop}{Proposition}
\newtheorem*{thm}{Theorem}
\newtheorem*{cor}{Corollary}
\newtheorem*{claim}{Claim}
\newtheorem*{axm}{Axiom}
\newtheorem {AXM}{Axiom}
\newtheorem*{lem}{Lemma}

\theoremstyle{remark}
\newtheorem*{rmk}{Remark}

\theoremstyle{definition}
\newtheorem*{define}{Definition}
\newtheorem*{eg}{Example}

\title[Construction of the $\bbZ$]
	{Analysis I Extension Lecture\\4. Construction of the Integers $\bbZ$}
\author{Asilata Bapat}
\begin{document}

\maketitle
\setuldepth{abcd}

\section*{Construction of $\bbZ$}
(Equivalence classes of pairs of naturals)
\newline
Let $Z = \bbN\times \bbN$. Define a relation $R$ on $Z \times Z$ ($R \subseteq Z \times Z$) as follows:
\begin{equation*}
R \vcentcolon = \Set{ \left( (a,b),(c,d) \right) \in Z \times Z) | a+d = b+c}
\end{equation*}
\begin{eg}
$\left( (3,1),(4,2) \right) \in R$ and $\left( (1,5),(5,9) \right) \in R$

\end{eg}
\begin{claim}
$R$ is an equivalence relations:
\end{claim}
\begin{enumerate}[(1)]
	\item (Reflexivity) 
		Since $a+b = b+a\quad \forall a,b \in \bbN$, we see that $\forall a, b \in \bbN$, the pair $((a,b), (a,b))$ in $R$.
	\item (Symmetry)
		If $((a,b), (c,d))\in R$ then $a+d = b+c$, so $c+a = d+b$ and so $((c,d), (a,b))\in R$
	\item (Transitivity)
		Suppose $((a+b),(c+d))\in R$ and $((c,d),(p,g)) \in R$. Then $a+b = b + c $ and $c + q = d + p$. After adding these and some manipulations, we see that 
		\begin{equation*}
			(a+q) + (c+d) = (b+p) + (c+d)
		\end{equation*}
		By cancellation , we have $a+q = b+ p$, so $((a,b),(p,q)) \in R$
\end{enumerate}
If $x \in Z$, write
\begin{equation*}
	[x] \vcentcolon = \Set{ y \in Z | (x,y) \in R }
\end{equation*}
Then $[x] \subseteq Z$ and $x \in [x]$, so $[x] \neq \varnothing$. This is the \ul{equivalence class} of $x$. $\bbZ$ is partitioned into disjoint, nonempty equivalence classes, so:
\begin{define}
	\begin{equation*}
	\begin{split} 
		\bbZ \vcentcolon = Z \bigg/ R &= \Set{ S \in \bbP(Z)| S = [x] \mbox{ for some } x \in Z}\\
		&=  \Set{\mbox{all equivalence classes of $R$}}
	\end{split}
	\end{equation*}
\end{define}
\begin{eg}
	$[(0,1)] = [(3,4)]$
\end{eg}

\par
There is an injective function $i:\bbN \mapsto \bbZ$ given by $n \mapsto [(n,0)]$.

\section*{The properties of $\bbZ$}
\begin{enumerate}[(1)]
	\item (Addition)
		$[(a,b)] + [(c,d)] \vcentcolon = [(a +_{\bbN} c, b +_{\bbN} d)]$ 
	\item (Negative)
		$-[(a,b)] \vcentcolon = [(b,a)]$, and $[(a,b)] + [(b,a,)] = [(0,0)]$
	\item (Subtraction)
		$[(a,b)] - [(c,d)] \vcentcolon = [(a,b)] + [(c,d)] = [(a+d,b+c)]$
	\item (Order Relation)
		We say $[(a,b)] < [(,d)]$ if $a + d < b + c$. Then this is a well-defined, total order. That is, if $[(a,b)]$ and $[(c,d)]$ are in $\bbZ$, then we have a trichotomy:
		\begin{equation*}
			[(a,b)] < [(c,d)] \mbox{ or } [(c,d)] < [(a,b)] \mbox{ or } [(a,b)] = [(c,d)]
		\end{equation*}

	\item (Multiplication)
		$[(a,b)] \cdot [(c,d)] \vcentcolon = [(ac+bd, ad + bc)]$
		\newline
		This is commutative, associative, and distributes over addition.
	\item (Absolute Value)
		\begin{equation*}
			|[a - b]| \vcentcolon = 
			\begin{cases}
				a - b \quad &\mbox{if } a \geq b\\
				b - a \quad &\mbox{otherwise}
			\end{cases}
		\end{equation*}
\end{enumerate}
Henceforth, we write $[(a,b)]$ as follows:
\begin{itemize}[-]
	\item
		If $a \geq b$, we replace it by $(a - b) \rightarrow$ Subtraction in $\bbN$
	\item
		If $a < b$, we replace it by $-(b - a ) \rightarrow$  subtraction in $\bbN$
\end{itemize}
\end{document}
