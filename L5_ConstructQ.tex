\documentclass[12pt]{amsart}

\usepackage[margin=1in]{geometry}
\usepackage{paralist}
\usepackage{amssymb}
\usepackage{amsmath}
\usepackage{amsthm}
\usepackage{braket}
\usepackage{mathtools}
\usepackage{soul}
\usepackage{parskip}

\newcommand{\bbR}{\mathbb{R}}
\newcommand{\bbN}{\mathbb{N}}
\newcommand{\bbZ}{\mathbb{Z}}
\newcommand{\bbQ}{\mathbb{Q}}
\newcommand{\bbP}{\mathbb{P}}
\newcommand{\bbC}{\mathbb{C}}
\newcommand{\suchthat}{\operatorname{\thinspace s.t. \thinspace}}
\newcommand{\ie}{\operatorname{\thinspace i.e. \thinspace}}
\newcommand{\LL}{\operatorname{LL}}

\theoremstyle{plain}
\newtheorem*{prop}{Proposition}
\newtheorem*{thm}{Theorem}
\newtheorem*{cor}{Corollary}
\newtheorem*{claim}{Claim}
\newtheorem*{axm}{Axiom}
\newtheorem {AXM}{Axiom}
\newtheorem*{lem}{Lemma}

\theoremstyle{remark}
\newtheorem*{rmk}{Remark}

\theoremstyle{definition}
\newtheorem*{define}{Definition}
\newtheorem*{eg}{Example}

\title[Construction of the Rational $\bbQ$]
	{Analysis I Extension Lectur\\5. Construction of the Rational $\bbQ$}
\author{Asilata Bapat}
\begin{document}

\maketitle
\setuldepth{document}

The construction of the rational $\bbR$ is very similar to that of of the Integers $\bbZ$.
\begin{define}
	We let $Q$ be equivalence classes of ordered pairs of (certain) integers:
	\begin{equation*}
		Q \vcentcolon = \Set{(a,b) \in \bbZ \times \bbZ | b \neq 0}
	\end{equation*}
\end{define}

\begin{rmk}
	Remember, $0 = [(0,0)] \in \bbZ\dots$
\end{rmk}

\begin{define}
	$R\in Q \times Q$ is defined as follows:
	\begin{equation*}
		R = \Set{((a,b), (c,d)) \in Q \times Q | ad = bc}
	\end{equation*}
\end{define}
\begin{claim}
	$R$ is an equivalence relation
\end{claim}
\begin{proof}
	Exercise.
\end{proof}

\begin{define}
	The set of rationals, $\ie \bbQ$, as 
	\begin{equation*}
		\bbQ \vcentcolon = Q\bigg/ R= \Set{\mbox{equivalence classes of parts of integers } (a,b) \mbox{ with } b \neq 0}
	\end{equation*}
\end{define}

\par
\ul{Operations/Properties} of $\bbQ$ 
\begin{enumerate}
	\item 
		Write $0 = [(0,1)]$ and $1 = [(1,1,)]$, and see that $0\neq 1$.
	\item (Addition)
		\begin{equation*}
		[(a,b)] +_{\bbQ} [(c,d)] \vcentcolon = [(ad +_{\bbZ} bc, bd)]
		\end{equation*}
	\item (Multiplication)
		\begin{equation*}
			[(a,b)]\cdot [(b,c)] \vcentcolon = [(ac,bd)]
		\end{equation*}
		This is commutative, associative, and distributes over addition.
	\item (Additive Inverse)
		For any $[(a,b)]\in \bbQ$, there is some $[(c,d)] \in \bbQ \suchthat$ 
		\begin{equation*}
		[(a,b)] + [(c,d)] = 0 
		\end{equation*}
		In fact, $[(a,b)] = [(-c,d)]$
	\item (Multiplication Inverse)
		If $[(a,b)] \neq 0$ then there exists a $[(c,d)] \neq 0 \suchthat$ 
		\begin{equation*}
			[(a,b)] \cdot [(c,d)] = 1
		\end{equation*}
		In fact $[(c,d)] = [(b,a)]$. 
	\item
		$0$ is the additive identity and $1$ is the multiplicative identity
\end{enumerate}

\begin{define}[Field]
	Any set $S$ together with elements $0,1 \in S$ and operations Addition($+$) and Multiplication($\cdot$): $S \times S \mapsto S$ satisfying the properties above is called a \ul{field}. So $\bbQ$ is a field.
\end{define}

\par
There are some further properties of $\bbQ$:
\begin{enumerate}[(1)]
	\item (Order)
		Say $[(a,b)]\in \bbQ$. We say that $[(a,b)] > 0$ if 
		\begin{equation*}
			b > 0 \mbox{ and } a > 0 \mbox{, \ul{or} } b < 0 \mbox{ and } a < 0
		\end{equation*}
		This is a total order on $\bbQ$ and satisfies trichotomy.
	\item (Distance)
		Write $Q^{\geqslant 0}$ to be the set of rationals that are either $0$ or positive. We have  function $d: \bbQ \times \bbQ \mapsto \bbQ^{\geqslant 0}$, defined as 
		\begin{equation*}
			d([(a,b)],[(c,d)]) = 
			\begin{cases}
				[(a,b)] - [(c,d)] \quad &\mbox{if } [(a,b)] \geqslant [(c,d)]\\	
				[(c,d)] - [(a,b)] \quad &\mbox{otherwise}
			\end{cases}
		\end{equation*}
		This function is a \ul{metric} in the following sense:
		\begin{enumerate}[a.]
			\item $d(x,y)\geqslant 0 \quad \forall x,y \mbox{ in } \bbQ$
			\item $d(x,y) = 0$ iff $x = y \quad \forall x,y \in \bbQ$ 
			\item $d(x,y) = d(y,x)$, and 
			\item $d(x,z) \leqslant d(x,y) + d(y,z)$
		\end{enumerate}
\end{enumerate}

\end{document}
