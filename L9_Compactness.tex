\documentclass[12pt]{amsart}

\usepackage[margin=1in]{geometry}
\usepackage{paralist}
\usepackage{amssymb}
\usepackage{amsmath}
\usepackage{amsthm}
\usepackage{braket}
\usepackage{mathtools}
\usepackage{soul}
\usepackage{parskip}

\newcommand{\bbR}{\mathbb{R}}
\newcommand{\bbN}{\mathbb{N}}
\newcommand{\bbZ}{\mathbb{Z}}
\newcommand{\bbQ}{\mathbb{Q}}
\newcommand{\bbP}{\mathbb{P}}
\newcommand{\bbC}{\mathbb{C}}
\newcommand{\suchthat}{\operatorname{\thinspace s.t. \thinspace}}
\newcommand{\ie}{\operatorname{\thinspace i.e. \thinspace}}
\newcommand{\LL}{\operatorname{LL}}

\theoremstyle{plain}
\newtheorem*{prop}{Proposition}
\newtheorem*{thm}{Theorem}
\newtheorem*{cor}{Corollary}
\newtheorem*{claim}{Claim}
\newtheorem*{axm}{Axiom}
\newtheorem {AXM}{Axiom}
\newtheorem*{lem}{Lemma}

\theoremstyle{remark}
\newtheorem*{rmk}{Remark}

\theoremstyle{definition}
\newtheorem*{define}{Definition}
\newtheorem*{eg}{Example}

\title[Compactness]
	{Analysis I Extension Lectur\\9. Compactness}
\author{Asilata Bapat}
\begin{document}
\maketitle

Compactness is a property (like connectedness) that some topological spaces have but others don't.
\begin{define}[Open Cover]
An open cover of a space $X$ is a collection of open sets whose union is $X$.
\end{define}

\begin{define}[Connectness]
A topological space $X$ is called compact if \underline{every} open cover of $X$ has a \underline{finite} subcover.
\end{define}

\begin{eg}
$\bbR$ is not compact. Consider the open covers $(n, n+2)$ for $n \in \bbZ$.
\end{eg}
\begin{rmk}
If $Y \subseteq X$, sometimes we specify an open cover of $Y$ by giving a set $\Set{u_{\alpha}}_{\alpha \in I}$ of open sets of $X$,whose union contains $Y$.
\end{rmk}

This suggests that a noncompact space is one that can ``escape out to infinity''. But seemingly finite spaces can also be noncompact. 

\begin{eg}
$(0,1)$ is not compact; consider the open cover $(\frac{1}{n}, \frac{1}{n+2})$ for $n \in \bbN$ and $n \geqslant 1$. This has no finite subcover.
\end{eg}

\begin{thm}
In the metric topology on $\bbR$, any closed interval $[a, b]$ is compact.
\end{thm}

\begin{proof}
If $a = b$, then $[a, b] = {a}$; this is compact because any finite set is compact.
\par
If $a \neq b$, consider an open cover $\{u_{\alpha}\}_{\alpha \in I}$ of $[a, b]$. Let $a \in u_{\alpha}$ for some $\alpha$. Since $u_{\alpha}$ is open, $\exists$ some $c \leqslant b \suchthat [a, c] \subseteq u_{\alpha}$. If $c = b$, then we are done; Otherwise, let 
\begin{equation*}
C = \Set{x \in (a, b] | [a, x] \text{ is contained in a finite union of the sets } u_{\beta}}
\end{equation*}
Notice that $c \in C$ so $C \neq \varnothing$. Clearly $C$ is bounded above by $b$. Let $L = \sup(C)$.

\begin{proof}[Pf: $L \in C$]
\hfill
\newline
Let $L \in u_{\beta}$, and suppose $L\notin C$. Then $[a, L]$ cannot be covered by finitely many sets $u_{\beta} \in \Set{u_{\alpha}}_{\alpha \in I}$. But if $L \in u_{\beta}$ and $u_{\beta}$ is open, there is some $\varepsilon \suchthat [L - \varepsilon, L] \subseteq u_{\beta}$. Since $L = \sup(C)$, $L-\varepsilon$ is not an upper bound of $C$, so $[a, L-\varepsilon]$ is contained in a finite union of sets $u_{\beta}$. So $[a, L]$ is contained in the above union of $u_{\beta}$, still finite. Contradiction.
\newline
$\therefore L \in C$
\end{proof}

\begin{proof}[Pf: $L=b$]
\hfill
\newline
Similar argument: If $L \in u_{\beta}$ and $L \neq b$, then $\exists \varepsilon > 0 \suchthat [L, L+\varepsilon] \subseteq u_{\beta}$. Then $L + \varepsilon \in C$, contradiction.
\newline
$\therefore L = b$.
\end{proof}
Therefore, $[a, b]$ is compact in $\bbR$.
\end{proof}

\begin{prop}
A closed subset of a compact space is compact in the subspace topology.
\end{prop}
\begin{proof}
Let $Y \subseteq X$, where $X$ is compact and $Y$ is closed. Consider any open cover of $Y$, written as $\Set{u_{\alpha} \cap Y}_{\alpha \in I}$ where $u_{\alpha}$ is open in $X$. Consider the cover $\{u_{\alpha}\}_{\alpha \in I} \cup (X-Y)$ which is an open cover in $X$ as $Y$ is closed. Since $X$ is compact, it has a finite subcover, say $v_1$, $v_2$, \dots $v_n$. Then 
\begin{equation*}
\{v_1\cap Y,v_2\cap Y,\dots v_n\cap Y\}
\end{equation*}
 is a finite subcover of the original open cover of $Y$.
\newline
$\therefore Y$ is also a compact set. 
\end{proof}

\begin{prop}
If $f:X\mapsto Y$ is continuous and surjective, and $X$ is compact, then $Y$ also is compact.
\end{prop}
\begin{proof}
Let $Y = \bigcup\limits_{\alpha \in I} u_{\alpha}$. Then 
\begin{equation*}
X = f^{-1}\left(\bigcup\limits_{\alpha \in I} u_{\alpha}\right) = \bigcup\limits_{\alpha \in I}f^{-1}(u_{\alpha})
\end{equation*}
Since $f$ is continuous and surjective.
This has a finite subcover, say $f^{-1}(v_1),f^{-1}(v_2),\dots ,f^{-1}(v_n)$. Then 
\begin{equation*}
\{v_1, v_2, \dots v_n\}
\end{equation*}
is a finite open cover of $Y$. $\therefore  Y$ is compact.
\end{proof}
\begin{eg}
A circle is compact since it is the continuous image of $[0, 1]$.
\end{eg}

\begin{prop}
If $X$ and $Y$ are compact, then $X \times Y$ is compact.
\end{prop}
\begin{proof}
Let $\bigcup\limits_{\alpha \in I} u_{\alpha}$ be open cover of $X  \times Y $. If $(x, y) \in X \times Y$, then $(x, y) \in u_{\alpha}$ for some $\alpha \in I$. Therefore $\exists$ some $A_{xy} \times B_{xy}, \mbox{ containing } (x,y)$, lying in $u_{\alpha}$. 
Fix $x\in X$, and consider $\Set{B_{xy}|y \in Y}$, which is a open cover of $Y$.
\par
Take a finite subcover, say $\Set{B_{xy_1},B_{xy_2}, \dots, B_{xy_n}}$. Set $A_x \vcentcolon = \bigcap\limits_{i=1}^{n}B_{xy_i}$. Then 
\begin{equation*}
A_x \times B_{xy_1},A_x \times B_{xy_2},\dots,A_x \times B_{xy_n}
\end{equation*}
cover $A_x \times Y$, and each $A_x \times B_{xy_i}$ is contained in some $u_{\alpha}$.
\par
Now consider $\Set{A_x|x\in X}$, which has a finite cover, say $A_{x_1},A_{x_2}, \dots, A_{x_m}$. Then $\Set{A_{x_i}}$ cover $X \times Y$, so the corresponding $u_{\alpha} 's$ cover $X \times Y$.
\end{proof}

\begin{thm}[Heine-Borel]
A subset of $\bbR^n$ is compact iff it is closed and bounded.
\end{thm}
\begin{proof}
Let $X \subseteq \bbR^n$ be closed and bounded. Then $\exists r \in \bbR \suchthat X\subseteq [-r,r] \times [-r,r] \dots \times [-r,r]$, which is compact. 
\newline
As $X$ is closed in an compact subset,it therefore is compact.
\newline
Now suppose $X$ is compact.
\begin{itemize}
\item[{\bfseries $X$ is bounded}:] 
Consider the open ball of radius $n \in \bbN$ around $0$ $B(n,0)$. Then
\begin{equation*}
\bigcup\limits_{n\in\bbN}B(n,0) = \bbR
\end{equation*}
Since $X$ is compact, there exists a finite sequence
\begin{equation*}
(n_i)_{i=1}^k \suchthat X \subseteq \bigcup\limits_{(n_i)}B(n_i, 0)
\end{equation*}
$\therefore \max({n_i})$ is an upper bound of $X$, implying $X$ is bounded.
\item[{\bfseries $X$ is closed}:] 
Assume, for sake of contradiction, $X$ is compact but not closed.
\newline
Let $x$ be a limit point of $X \suchthat x \notin X$. Accordingly $\forall r > 0, \overline{B}(r, x) \cap X \neq \varnothing$. Then
$\Set{\bbR^n - \overline{B}(r,x)}$ is an open cover of $X$. That is,
\begin{equation*}
\bigcup\limits_{r\in \bbR}\left( \bbR^n - \overline{B}(r, x) \right) = \bbR - \{x\} \supseteq X
\end{equation*}
By compactness of $X$, there exists finite $\{r_i\}_{1\leqslant i\leqslant n} \suchthat X \subseteq \bigcup\limits_{i=1}^n\left( \bbR^n-\overline{B}(r_i,x) \right) = Y$.
\newline
However, it is easy to verify that $Y \cap B(\min(r_i), x) = \varnothing$, contradicting with assumption that $x$ is a limit point of $X$.
\newline
Therefore, $X$ is closed.
\end{itemize}
\end{proof}
\end{document}
