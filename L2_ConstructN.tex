\documentclass[12pt]{amsart}

\usepackage[margin=1in]{geometry}
\usepackage{paralist}
\usepackage{amssymb}
\usepackage{amsmath}
\usepackage{amsthm}
\usepackage{braket}
\usepackage{mathtools}
\usepackage{soul}
\usepackage{parskip}

\newcommand{\bbR}{\mathbb{R}}
\newcommand{\bbN}{\mathbb{N}}
\newcommand{\bbZ}{\mathbb{Z}}
\newcommand{\bbQ}{\mathbb{Q}}
\newcommand{\bbP}{\mathbb{P}}
\newcommand{\bbC}{\mathbb{C}}
\newcommand{\suchthat}{\operatorname{\thinspace s.t. \thinspace}}
\newcommand{\ie}{\operatorname{\thinspace i.e. \thinspace}}
\newcommand{\LL}{\operatorname{LL}}

\theoremstyle{plain}
\newtheorem*{prop}{Proposition}
\newtheorem*{thm}{Theorem}
\newtheorem*{cor}{Corollary}
\newtheorem*{claim}{Claim}
\newtheorem*{axm}{Axiom}
\newtheorem {AXM}{Axiom}
\newtheorem*{lem}{Lemma}

\theoremstyle{remark}
\newtheorem*{rmk}{Remark}

\theoremstyle{definition}
\newtheorem*{define}{Definition}
\newtheorem*{eg}{Example}

\title[Construction of $\bbN$]
	{Analysis I Extension Lecture\\2. Construction of $\bbN$}
\author{Asilata Bapat}
\begin{document}
\maketitle
\setuldepth{abcd}

\begin{rmk}
	Remember: In this world, everything is strictly a set.
\end{rmk}

Start with $0$. We define $0 \vcentcolon = \varnothing$.
Recall $\operatorname{succ}(x) \vcentcolon = x \cup \Set{x}$. 
Accordingly,
\begin{align*}
	1 &\vcentcolon = \varnothing \cup \Set{\varnothing} = \operatorname{succ}(\varnothing) = \Set{\varnothing}\\
	2 &\vcentcolon = \operatorname{succ}(1) = \operatorname{succ}(\Set{\varnothing}) = \Set{\varnothing} \cup \Set{\Set{\varnothing}} = \Set{\varnothing, \Set{\varnothing}}\\
	3 &\vcentcolon = \operatorname{succ}(2) = \Set{\varnothing, \Set{\varnothing}} \cup \Set{\Set{\varnothing, \Set{\varnothing}}} = \Set{\varnothing, \Set{\varnothing}, \Set{\varnothing, \Set{\varnothing}}}\\
	\vdots
\end{align*}

\ul{But what is $\bbN$?} We can define $\bbN$ by using the Axiom of Infinity --- We want $\bbN$ to contain $0 = \varnothing$, successor of $0$, the successor of that, and so on, and nothing else. 
\newline

Let $S$ be a set such that $\varnothing \in S$ and if $x \in S$ then $\operatorname{sucss}(x) \in S$. However, $S$ \ul{could} be a lot bigger than $\bbN$, so we have to do some more work. 

\begin{define}
	\begin{align*}
		I_S &= \Set{T \in \mathbb{P}(S)| \varnothing \in T \mbox{ and } x \in T,\operatorname{succ}(x) \in S}\\
		\bbN &= \Set{x \in S | \forall T \in I_S, x \in T} =  \bigcup\limits_{u \in I_S} u
	\end{align*}
\end{define}
That is $I_S$ is set of all inductive subset of $S$. $I_S \neq \varnothing$ because $S \in I_S$.

\begin{thm}[Principle of Mathmetical Induction]
	Let $p$ be a predicate (function that returns TRUE/FALSE defined on $\bbN$.
		Assume that $p(0)$ is true and $\forall k \in \bbN$, $p(k) \implies p(\operatorname{succ}(k))$, then $p(n)$ holds for all $n \in \bbN$.
\end{thm}

\begin{proof}
	Fix $p$, with the above properties, and set $S \vcentcolon = \Set{n \in \bbN | p(n)}$. We want to show $S = \bbN$. $\ie$ the element of $S$ are exactly the element of $\bbN$.
	\newline
	We observe that $S$ is inductive. Specifically,
	\begin{enumerate}[(1)]
		\item $0 \in S$ because $p(0) = p(\varnothing)$ holds.
		\item If $x \in S$, it means $p(x)$ holds. But $p$ has the property that $p(x) \implies p(x^+)$. Then by definition of $S$, $x^+ \in S$.
	\end{enumerate}
	By $(1)$ and $(2)$, $S$ is inductive, and therefore $\bbN \subseteq S$.
	By the definition of $S$, it is a subset of $\bbN$, therefore $S \subseteq \bbN$.
	\newline
	$\therefore S = N$ 
\end{proof}

\begin{thm}
	If $m, n$ are two natural numbers such that $m^+ = n^+$, then $m = n$.
\end{thm}

\begin{lem}
	Let $x, n \in \bbN$. If $x \in n$, then $x \subset n$.
\end{lem}
\begin{proof}
	Define $p(n)$ as $p(n) = \forall x(x\in n \implies x \subset n)$
\end{proof}
\end{document}
