\documentclass[12pt]{amsart}

\usepackage[margin=1in]{geometry}
\usepackage{paralist}
\usepackage{amssymb}
\usepackage{amsmath}
\usepackage{amsthm}
\usepackage{braket}
\usepackage{mathtools}
\usepackage{soul}
\usepackage{parskip}

\newcommand{\bbR}{\mathbb{R}}
\newcommand{\bbN}{\mathbb{N}}
\newcommand{\bbZ}{\mathbb{Z}}
\newcommand{\bbQ}{\mathbb{Q}}
\newcommand{\bbP}{\mathbb{P}}
\newcommand{\bbC}{\mathbb{C}}
\newcommand{\suchthat}{\operatorname{\thinspace s.t. \thinspace}}
\newcommand{\ie}{\operatorname{\thinspace i.e. \thinspace}}
\newcommand{\LL}{\operatorname{LL}}

\theoremstyle{plain}
\newtheorem*{prop}{Proposition}
\newtheorem*{thm}{Theorem}
\newtheorem*{cor}{Corollary}
\newtheorem*{claim}{Claim}
\newtheorem*{axm}{Axiom}
\newtheorem {AXM}{Axiom}
\newtheorem*{lem}{Lemma}

\theoremstyle{remark}
\newtheorem*{rmk}{Remark}

\theoremstyle{definition}
\newtheorem*{define}{Definition}
\newtheorem*{eg}{Example}

\title[Connectedness]
	{Connectedness and Pass-Conectedness}
\author{Asilata Bapat}
\begin{document}

\maketitle
\setuldepth{abcd}

\section*{Connectedness}
\begin{define}
	A topological space $X$ is connected if it cannot be expressed as the union of two non-empty, disjoint open sets.
\end{define}
\begin{eg}
	$X = [0, 1] \cup (3,4)$ is disconnected (subspace topology).
	\newline
	$[0,1]$ is open because $[0, 1] = (-\frac{1}{2}, \frac{1}{2} + 1)$ is intersection of $X$ is intersection of $X$ and an open set in $\bbR$. Therefore $X$ is union of $[0,1]$ and $(3,4)$, which are both open and mutually disjoint.
\end{eg}

Equivalently, $X$ is connected if it cannot be expressed as the union of two nonempty, disjoint closed sets. Equivalently, $X$ is connected iff the only subsets of $X$ that are both open and closed are $\varnothing$ and $X$.

\begin{eg}
	$\bbQ \subseteq \bbR$ is disconnected, because 
	\begin{equation*}
		\bbQ = (\bbQ \cap (-\infty, \sqrt{2})) \cup (\bbQ \cap (\sqrt{2}, \infty))
	\end{equation*}
\end{eg}

\begin{rmk}
	Is $[0, 1]$ connected in $\bbR_{\text{metric}}$? What about $(0, 1)$?
\end{rmk}

\begin{thm}
	Every single interval (open, closed, or half-open) is connected in $\bbR_{\text{metric}}$
\end{thm}

\begin{proof}
	We will prove it for closed intervals for simplicity.
	\newline
	Let $X = [a, b]$. If $a = b$, $X$ is a singleton, so is connected. Suppose $a \neq b$ and $a < b$. Suppose that $X = A \cup B$, where $A$ and $B$ are both clopen and $A$ is not empty. We need to show $B$ has to be $\varnothing$. 
	\par
	Let's also suppose, WLOG, $a \in A$. First of all, $A$ is open in $X$, so 
	\begin{equation*}
		A = [a, b] \cap U \text{ , where } U \text{ is open in } \bbR
	\end{equation*}
	Also, $a \in A \implies a \in U$. Since $U$ is open, $\exists \varepsilon $ sufficiently small $\suchthat (a-\varepsilon, a+\varepsilon) \subseteq U$.
	Then $(a-\varepsilon, a+\varepsilon) \cap [a, b] \subseteq A \implies [a, a+\varepsilon) \subseteq A$.
	\newline
	Define the set 
	\begin{equation*}
		C = \Set{c \in [a, b]| [a, c] \subseteq A}
	\end{equation*}
	Clearly $C$ has an upper bound, namely $b$. Therefore $c$ has a least upper bound $L \leqslant b$.
	\par
	Now we are going to show that $L \in C$.
	\newline
	$L$ is a limit point of $A$. $L \in X$ because $a < L \leqslant b$. $A$ is clopen in $X$ so $A$ is closed in $X$, so $L \in A \implies [a, L] \subseteq A$.
	\par
	Now we show that $L = b$. Suppose, for contradiction,  $L < b$.
	As $L\in A$ and $A$ is open in $X$, $\exists \varepsilon > 0 \suchthat [L, \varepsilon) \subseteq A$. Therefore,$[L, \varepsilon] \subseteq A$ since $A$ is closed, indicating $L + \varepsilon \in C$. However, $L = \sup(C)$. Contradiction. 
	$\therefore L = b$.
	Therefore, the whole proof implies $A = X$ and $B = \varnothing$. Accordingly,  any closed interval is connected in $R_{\text{metric}}$.
\end{proof}

\section*{Path-connectedness}
Now that we know that $[0,1]$ is connected, we use it to define another notion.

\begin{define}[Path-Connectedness]
	A space $X$ is path-connected if $\forall a,b \in X$, there is a ``path'' from $a$ to $b$. That is, if there is a continuous function $f:[0,1] \mapsto X$ with $f(0) = a$ and $f(1) = b$.
\end{define}

\begin{rmk}
	$[0, 1]$ is a ``stand-in'' for any other closed interval because $[0,1]$ and $[a, b]$ are homeomorphic for any $p \neq q$. (Exercise)
\end{rmk}

\begin{prop}
	If $X$ is path-connected, then it is connected.
\end{prop}

\begin{proof}
	Suppose for contradiction, that $X$ is path-connected but not connected so $X = A \cup B$, $A \neq \varnothing$, $B \neq \varnothing$, $A \cap B = \varnothing$, and $A,B$ is open. Let $a \in A$ and $b \in B$. There is some $f:[0,1] \implies X \suchthat$
	\begin{equation*}
		f(0) = a \text{ and } f(1) = b 
	\end{equation*}
	So, consider $f^{-1}(A)$ and $f^{-1}(B)$. Not that $f^{-1}(A)$ and $f^{-1}(B)$ are open, because $f$ is continuous. They are also nonempty and disjoint (because if $x\in f^{-1}(A)$ and if $x \in f^{-1}(B)$, then $f(x) \in A \cup B$).
	\par
	Also, $f^{-1}(A) \cup f^{-1}(B) = [0, 1]$. This contradicts the fact that $[0,1]j$ is connected.
\end{proof}

This gives an easy chance to see if spaces are connected.

\begin{eg}
	\hfill
	\begin{itemize}
		\item 
			Circle in $\bbR^2$ is connected.
		\item
			$\forall n\in \bbN^{+}$, $\bbR^n$ is connected.
		\item
			Any interval is connected because it is path connected. We can consider
			\begin{equation*}
				f(t) \vcentcolon = (1-t)a + tb \quad\quad(0\leqslant t \leqslant 1)
			\end{equation*}

	\end{itemize}
	
\end{eg}

\begin{prop}
	If $X \subseteq \bbR^n$ is \ul{convex}, it is connected.
\end{prop}

\begin{rmk}
	There exist connected spaces that are not path-connected! A counterexample is sine curve,
	\begin{equation*}
		\Set{\left(x,\sin\left(\frac{1}{x}\right)\right)| x\in (0, 1]} \cup \Set{(0,0)}
	\end{equation*}
\end{rmk}

\begin{prop}
	Let $f: X \mapsto Y$ be continuous and surjective. Then:
	\begin{enumerate}[(1)]
		\item If $X$ is connected, so is $Y$.
		\item If $X$ is path-connected, so is $Y$.
	\end{enumerate}
\end{prop}

Now we can compare spaces and distinguish them.

\begin{cor}
	If $X$ and $Y$ are homeomorphic, then 
	\begin{enumerate}[(1)]
		\item X is connected iff Y is connected.
		\item X is path-connected iff Y is path-connected.
	\end{enumerate}
\end{cor}

This can be used to deduce (for example) that $\bbR$ is not homeomorphic to $\bbR^n$ for $n > 1$.

\begin{define}[Connected Component]
	Let $X$ be a space and $x \in X$. The connected component of $x \in X$ is the union of all connected subspace of $X$ containing $x$, denoted by $C_x$. Especially, $X$ is itself connected.
\end{define}

\begin{define}[Path Components]
	Let X be a space and $x \in X$. The path component of $x$ is the set of all $y\in X \suchthat$ there is a path from $x$ to $y$, called $P_x$.
\end{define}

\begin{rmk}
	Say $x {\sim}_{c} y$ if $C_x = C_y$; $x {\sim}_{p} y$ if $P_x = P_y$. Then these are equivalent relations.
\end{rmk}

\begin{prop}
	Every path-connected space is connected. In particular, every interval in $\bbR$ is connected.
\end{prop}

These notions let us distinguish spaces from each others. Recall:

\begin{thm}
	Let $f:X\mapsto Y$ be continuous and surjective, 
\begin{enumerate}[(1)]
\item If $X$ is connected, then $Y$ is connected.
\item If $X$ is path-connected, then $Y$ is path connected. 
\end{enumerate}

In particular, if $f:X\mapsto Y$ is a homeomorphism, then $X$ is connected (resp. path-connected) iff $Y$ is connected (resp. path-connected)
\end{thm}

\begin{prop}
$\bbR$ is not homeomorphic to $\bbR^n$ for any $n > 1$
\end{prop}

\begin{proof}
Suppose $f:\bbR\mapsto \bbR^n$ is a homeomorphism. Then $f:\bbR - \Set{0} \mapsto \bbR^n - \Set{f(0)}$ is a homeomorphism.
$\bbR - \Set{0}$ is neither connected nor path-connected. But $\bbR^n - \Set{f(0)}$ is path-connected. Contradiction. 
\end{proof}
\end{document}
